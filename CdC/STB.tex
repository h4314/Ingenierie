% generated by Docutils <http://docutils.sourceforge.net/>
\documentclass[a4paper,12pt]{article}
\usepackage{fixltx2e} % LaTeX patches, \textsubscript
\usepackage{cmap} % fix search and cut-and-paste in PDF
\usepackage[french]{babel}
\usepackage[T1]{fontenc}
\usepackage[utf8]{inputenc}
\usepackage{ifthen}
\usepackage{color}

%%% User specified packages and stylesheets
\usepackage{include}

%%% Fallback definitions for Docutils-specific commands

% admonition (specially marked topic)
\providecommand{\DUadmonition}[2][class-arg]{%
  % try \DUadmonition#1{#2}:
  \ifcsname DUadmonition#1\endcsname%
    \csname DUadmonition#1\endcsname{#2}%
  \else
    \begin{center}
      \fbox{\parbox{0.9\textwidth}{#2}}
    \end{center}
  \fi
}

% error admonition title
\providecommand*{\DUtitleerror}[1]{\DUtitle{\color{red}#1}}

% title for topics, admonitions and sidebar
\providecommand*{\DUtitle}[2][class-arg]{%
  % call \DUtitle#1{#2} if it exists:
  \ifcsname DUtitle#1\endcsname%
    \csname DUtitle#1\endcsname{#2}%
  \else
    \smallskip\noindent\textbf{#2}\smallskip%
  \fi
}

% hyperlinks:
\ifthenelse{\isundefined{\hypersetup}}{
  \usepackage[colorlinks=true,linkcolor=blue,urlcolor=blue]{hyperref}
}{}
\hypersetup{
  pdftitle={Spécifications Techniques des Besoins},
}

%%% Body
\begin{document}

% Document title
\title{Spécifications Techniques des Besoins%
  \phantomsection%
  \label{specifications-techniques-des-besoins}}
\author{Hexanôme H4314}
\date{}
\maketitle

\DUadmonition[system-message]{
\DUtitle[system-message]{system-message}


{\color{red}WARNING/2} in \texttt{STB.rst}, line~1

Title overline too short.
%
\begin{quote}{\ttfamily \raggedright \noindent
====================================\\
Spécifications~Techniques~des~Besoins\\
====================================
}
\end{quote}
backrefs: }


%___________________________________________________________________________

\newpage \section*{1. Axes d'amélioration%
  \phantomsection%
  \addcontentsline{toc}{section}{1. Axes d'amélioration}%
  \label{axes-d-amelioration}%
}

Ce document a pour objectif de détailler les besoins du nouveau système. Ces exigences sont basées sur les axes d'améliorations provenant de l'analyse de l'existant.

L'objectif premier est d'effectuer des économies en termes de coûts directs car le système doit être appliquer massivement à travers l'Europe.
%
\begin{quote}
%
\begin{itemize}

\item Centralisation de la surveillance : économies d'échelle.

\item Une meilleure logistique, notamment en termes de transport.

\item Des ressources humaines mieux utilisées. Les opérationnels doivent passer moins de temps à la surveillance (faible valeur ajoutée) pour se concentrer sur leur métier.

\end{itemize}

\end{quote}

Un deuxième objectif se situe aux niveau de la des réductions de coûts indirects par une qualité de surveillance accrue.
%
\begin{quote}
%
\begin{itemize}

\item Un meilleur contrôle des risques environnementaux

\item Un gaspillage des ressources réduit au minimum (énergie, déchets, contenant des cuves)

\item Anticipation des problèmes

\item Automatisation de la surveillance, fiabilité augmentée

\item Une meilleure traçabilité des opérations

\item Une aide à la décision à partir des données collectées

\end{itemize}

\end{quote}


%___________________________________________________________________________

\newpage \section*{2. Exigences fonctionnelles%
  \phantomsection%
  \addcontentsline{toc}{section}{2. Exigences fonctionnelles}%
  \label{exigences-fonctionnelles}%
}


%___________________________________________________________________________

\subsection*{FO-1 Monitoring à distance%
  \phantomsection%
  \addcontentsline{toc}{subsection}{FO-1 Monitoring à distance}%
  \label{fo-1-monitoring-a-distance}%
}

Un des enjeux majeur de ce nouveau système est sa capacité à surveiller à distance l'état des sites. Cette surveillance sera donc possible à partir du site central mais aussi via d'autres plateformes décentralisées que ce soit des postes fixes ou des appareils mobiles de type smartphone utiles pour les opérationnels.


%___________________________________________________________________________

\subsubsection*{FO-1-a Monitoring de l'état des cuves%
  \phantomsection%
  \addcontentsline{toc}{subsubsection}{FO-1-a Monitoring de l'état des cuves}%
  \label{fo-1-a-monitoring-de-l-etat-des-cuves}%
}

Il est possible de connaître l'état de chaque cuve d'une station distante, c'est à dire d'être capable de lire en temps réel les valeurs de chacun de ses capteurs. Les lectures de données doivent être facilitées par la mise à disposition de valeurs de références : valeur minimale, valeur maximale, seuils d'alertes, type de mesure, date de la dernière mise à jour, etc.


%___________________________________________________________________________

\subsubsection*{FO-1-b Monitoring des anomalies%
  \phantomsection%
  \addcontentsline{toc}{subsubsection}{FO-1-b Monitoring des anomalies}%
  \label{fo-1-b-monitoring-des-anomalies}%
}

Les sites distants doivent informer le site central de toute anomalie encourue. Les anomalies peuvent être liées à différents types de dysfonctionnement :
%
\begin{quote}
%
\begin{itemize}

\item un capteur ne fonctionne plus

\item un capteur renseigne des valeurs incohérentes

\item une cuve ne transmet plus d'informations au nœud maître

\end{itemize}

\end{quote}

De plus le site central doit être capable de détecter une perte de communication avec un site distant.


%___________________________________________________________________________

\subsubsection*{FO-1-c Localisation géographique%
  \phantomsection%
  \addcontentsline{toc}{subsubsection}{FO-1-c Localisation géographique}%
  \label{fo-1-c-localisation-geographique}%
}

Les stations doivent être localisables géographiquement tout comme les cuves. La majorité des cuves étant immobiles, cette fonctionnalité sera facile à mettre en place. Cependant certaines cuves pourraient potentiellement être mobiles et ce cas devrait donc être pris en compte.


%___________________________________________________________________________

\subsection*{FO-2 Maintenance à distance%
  \phantomsection%
  \addcontentsline{toc}{subsection}{FO-2 Maintenance à distance}%
  \label{fo-2-maintenance-a-distance}%
}

La partie logicielle doit pouvoir être maintenue depuis le site central, c'est à dire qu'il ne sera pas nécessaire d'envoyer du personnel pour effectuer une mise à jour du logiciel, le paramétrer ou bien pour analyser son fonctionnement.

Concernant la partie matérielle, bien qu'impossible d'être maintenue à distance, elle doit comporter des mécanismes d'alarmes et d'indicateurs qui peuvent tenir le site central informé de toute anomalie encourue ou bien lorsque des problèmes peuvent être anticipés (exemple : niveau de batterie faible).


%___________________________________________________________________________

\subsection*{FO-3 Maintenance sur site%
  \phantomsection%
  \addcontentsline{toc}{subsection}{FO-3 Maintenance sur site}%
  \label{fo-3-maintenance-sur-site}%
}

Chaque site doit posséder des moyens de connexion au système pour qu'un opérationnel puisse diriger le fonctionnement sur place. Il pourra à la fois lire les informations circulant sur le réseau interne (données des capteurs, état des cuves, état du réseau, etc.) et effectuer des opérations de maintenance : relance du système, paramétrage, mise à jour logicielle ou maintenance matérielle. Dans ce dernier cas il convient de rendre le matériel facilement accessible.


%___________________________________________________________________________

\subsection*{FO-4 Traitements sur site central%
  \phantomsection%
  \addcontentsline{toc}{subsection}{FO-4 Traitements sur site central}%
  \label{fo-4-traitements-sur-site-central}%
}


%___________________________________________________________________________

\subsubsection*{FO-4-a Agrégation des données provenant des sites centraux%
  \phantomsection%
  \addcontentsline{toc}{subsubsection}{FO-4-a Agrégation des données provenant des sites centraux}%
  \label{fo-4-a-agregation-des-donnees-provenant-des-sites-centraux}%
}

Le site central doit pouvoir stocker de manière structurée l'ensemble des données émises par les sites distants.


%___________________________________________________________________________

\subsubsection*{FO-4-b Planification des interventions%
  \phantomsection%
  \addcontentsline{toc}{subsubsection}{FO-4-b Planification des interventions}%
  \label{fo-4-b-planification-des-interventions}%
}

A partir des données relevées, le système doit pouvoir générer des plannings d'intervention. Les interventions peuvent être de type :
%
\begin{quote}
%
\begin{itemize}

\item affectation d'un opérationnel pour une maintenance logicielle ou matérielle

\item envoi d'un camion pour effectuer un ravitaillement ou un enlèvement sur un ou plusieurs sites

\end{itemize}

\end{quote}

L'objectif ici est d'optimiser la logistique, notamment grouper les opérations pour rentabiliser les déplacements au maximum : ravitailler plusieurs sites en un trajet et essayer d'optimiser les chargements d'un camion. Dans l'idéal un camion doit transporter sa charge maximale et l'utiliser totalement. Certains camions pourraient également être toujours pleins, que ce soit à l'aller et au retour dans le cas où les ravitaillement et enlèvement peuvent être compatibles.

La planification est donc basée sur la possibilité de définir des routes de maintenance entre diverses cuves et zones homogènes.

La planification inclut des sous fonctionnalités :
%
\begin{quote}
%
\begin{itemize}

\item Gestion des prestataires : fournisseurs externes

\item Gestion de la disponibilité des opérationnels

\end{itemize}

\end{quote}


%___________________________________________________________________________

\subsubsection*{FO-4-c Suivi en temps réel des interventions%
  \phantomsection%
  \addcontentsline{toc}{subsubsection}{FO-4-c Suivi en temps réel des interventions}%
  \label{fo-4-c-suivi-en-temps-reel-des-interventions}%
}

Les interventions doivent être localisables en temps réel : il est possible de connaître la position géographique exacte des opérationnels et des camions, ces derniers ayant bien souvent un système de balise GPS intégrée.


%___________________________________________________________________________

\subsubsection*{FO-4-d Aide à la décision%
  \phantomsection%
  \addcontentsline{toc}{subsubsection}{FO-4-d Aide à la décision}%
  \label{fo-4-d-aide-a-la-decision}%
}

L'objectif, ici, est d'utiliser les connaissances accumulées depuis la mise en service du système afin d'améliorer son efficacité : accroissement de la qualité de service et réduction des coûts. Il s'agit donc de générer des statistiques globales, par sites et par cuves sur lesquelles les opérationnelles pourront se baser pour prendre des décisions. Mais il s'agit d'aller plus loin : proposer aux opérationnels des axes d'améliorations en mettant en relief les dysfonctionnements, les goulets d'étranglement ainsi que des moyens d'anticiper les problèmes.


%___________________________________________________________________________

\subsection*{FO-5 Traitements sur station%
  \phantomsection%
  \addcontentsline{toc}{subsection}{FO-5 Traitements sur station}%
  \label{fo-5-traitements-sur-station}%
}


%___________________________________________________________________________

\subsubsection*{FO-5-a Relevé des capteurs%
  \phantomsection%
  \addcontentsline{toc}{subsubsection}{FO-5-a Relevé des capteurs}%
  \label{fo-5-a-releve-des-capteurs}%
}

Chaque cuve doit posséder au minimum un capteur d'acquisition de données lié au contenu de la cuve (exemple : niveau de profondeur).


%___________________________________________________________________________

\subsubsection*{FO-5-b Uniformisation des données%
  \phantomsection%
  \addcontentsline{toc}{subsubsection}{FO-5-b Uniformisation des données}%
  \label{fo-5-b-uniformisation-des-donnees}%
}

Les données relevées sur par les capteurs doivent être uniformisées sous un protocole fixé.


%___________________________________________________________________________

\subsubsection*{FO-5-c Circulation de l'information sur le réseau interne%
  \phantomsection%
  \addcontentsline{toc}{subsubsection}{FO-5-c Circulation de l'information sur le réseau interne}%
  \label{fo-5-c-circulation-de-l-information-sur-le-reseau-interne}%
}

Les données des noeuds esclaves doivent circuler de noeud en noeud jusqu'au noeud maître.


%___________________________________________________________________________

\subsubsection*{FO-5-d Communication de l'information vers le site central%
  \phantomsection%
  \addcontentsline{toc}{subsubsection}{FO-5-d Communication de l'information vers le site central}%
  \label{fo-5-d-communication-de-l-information-vers-le-site-central}%
}

Le noeud maître se chargera de transformer puis communiquer les données de l'ensemble des cuves de la station au site central.


%___________________________________________________________________________

\subsubsection*{FO-5-e Optimisation de la gestion de l'énergie%
  \phantomsection%
  \addcontentsline{toc}{subsubsection}{FO-5-e Optimisation de la gestion de l'énergie}%
  \label{fo-5-e-optimisation-de-la-gestion-de-l-energie}%
}

Chaque élément du site distant : capteur, noeud esclave et maître devront comprendre des moyens d'optimisation de l'énergie : mise en veille provisoire, etc.


%___________________________________________________________________________

\subsection*{FO-6 Traçabilité%
  \phantomsection%
  \addcontentsline{toc}{subsection}{FO-6 Traçabilité}%
  \label{fo-6-tracabilite}%
}

Le système doit archiver tout un ensemble d'évènements :
%
\begin{itemize}

\item arrivée d'informations provenant des sites distants : données de capteur, anomalies, rapport de maintenance sur site distant etc.

\item ensemble des actions prises par les différents acteurs : opérations de maintenance, ravitaillement/enlèvement du contenu d'une cuve, etc.

\end{itemize}


%___________________________________________________________________________

\newpage \section*{3. Priorisation des exigences fonctionnelles%
  \phantomsection%
  \addcontentsline{toc}{section}{3. Priorisation des exigences fonctionnelles}%
  \label{priorisation-des-exigences-fonctionnelles}%
}

Certaines fonctionnalités sont à prévoir immédiatement tandis que d'autres peuvent être considérées comme moins prioritaires car complexes ou impactant faiblement les axes d'améliorations. Il convient donc de les classer en deux lots.


%___________________________________________________________________________

\subsection*{Lot 1 - Besoins immédiats%
  \phantomsection%
  \addcontentsline{toc}{subsection}{Lot 1 - Besoins immédiats}%
  \label{lot-1-besoins-immediats}%
}

Ces fonctionnalités doivent être implémentées dès la première version du système car elles sont considérées vitales pour un fonctionnement global. De plus elles prennent en compte fortement les axes d'amélioration retenus.
%
\begin{quote}
%
\begin{itemize}

\item %
\begin{description}
\item[{FO-1 Monitoring à distance}] \leavevmode %
\begin{itemize}

\item FO-1-a Monitoring de l'état des cuves

\item FO-1-b Monitoring des anomalies

\end{itemize}

\end{description}

\item FO-2 Maintenance à distance

\item FO-3 Maintenance sur site

\item %
\begin{description}
\item[{FO-4 Traitements sur site central}] \leavevmode %
\begin{itemize}

\item FO-4-a Aggrégation des données provenant des sites centraux

\item FO-4-b Planification des interventions

\end{itemize}

\end{description}

\item %
\begin{description}
\item[{FO-5 Traitements sur station}] \leavevmode %
\begin{itemize}

\item FO-5-a Relevé des capteurs

\item FO-5-b Uniformisation des données

\item FO-5-c Circulation de l'information sur le réseau interne

\item FO-5-d Communication de l'information vers le site central

\end{itemize}

\end{description}

\item FO-6 Traçabilité

\end{itemize}

\end{quote}


%___________________________________________________________________________

\subsection*{Lot 2 - Besoins à long terme%
  \phantomsection%
  \addcontentsline{toc}{subsection}{Lot 2 - Besoins à long terme}%
  \label{lot-2-besoins-a-long-terme}%
}

Ces fonctionnalités pourront être implémentées dans un deuxième temps car elles ne sont pas vitales pour la création d'un système fonctionnel. Elles sont souvent complexes et donc longues à mettre en place. De plus elles ne répondent que faiblement aux axes d'amélioration.
%
\begin{quote}
%
\begin{itemize}

\item %
\begin{description}
\item[{FO-1 Monitoring à distance}] \leavevmode %
\begin{itemize}

\item FO-1-c Localisation géographique

\end{itemize}

\end{description}

\item %
\begin{description}
\item[{FO-4 Traitements sur site central}] \leavevmode %
\begin{itemize}

\item FO-4-c Suivi en temps réel des interventions

\item FO-4-d Aide à la décision

\end{itemize}

\end{description}

\item %
\begin{description}
\item[{FO-5 Traitements sur station}] \leavevmode %
\begin{itemize}

\item FO-5-e Optimiser la gestion de l'énergie

\end{itemize}

\end{description}

\end{itemize}

\end{quote}


%___________________________________________________________________________

\newpage \section*{4. Exigences non fonctionnelles%
  \phantomsection%
  \addcontentsline{toc}{section}{4. Exigences non fonctionnelles}%
  \label{exigences-non-fonctionnelles}%
}

Sont définis ci-après les besoins non fonctionnels classés par ordre d'importance. Ils proposent la mise en place de contraintes et de bonnes pratiques à appliquer sur les fonctionnalités énoncées ci-dessus. Dans un deuxième temps, il conviendra d'établir un tableau croisé des fonctionnalités et des contraintes non fonctionnelles à respecter pour chacune d'elles.


%___________________________________________________________________________

\subsection*{4.1. Intégration de l'existant%
  \phantomsection%
  \addcontentsline{toc}{subsection}{4.1. Intégration de l'existant}%
  \label{integration-de-l-existant}%
}

La réussite de ce projet passe par l'adhésion des exploitants chargés actuellement de la surveillance de ces sites ainsi que des prestataires. Il convient donc de comprendre comment ceux-ci fonctionnent et en quoi le nouveau système affectera leurs opérations. Il s'agira donc de faire participer ces personnes dans l'élaboration de la solution et de l'organisation fonctionnelle qui en découlera. Ces personnes doivent être convaincues de la nécessité d'un tel projet. Ils devront être régulièrement informés par des sessions de formations ainsi que par la mise à disposition de documentation.


%___________________________________________________________________________

\subsection*{4.2. Robustesse%
  \phantomsection%
  \addcontentsline{toc}{subsection}{4.2. Robustesse}%
  \label{robustesse}%
}

Les stations doit être protégées correctement pour supporter des conditions climatiques extrêmes dans certaines régions (température, humidité, vent, éclairs, etc.). Le système embarqué doit pouvoir revenir à un état stable en cas de reprise. Côté site central, le stockage des données est conséquent et devra se munir de systèmes préventifs pour éviter toute perte de données : par exemple en duplicant les données sur des systèmes externes. Les systèmes embarqués doivent être quasiment ou complètement énergétiquement autonomes.


%___________________________________________________________________________

\subsection*{4.3. Fiabilité%
  \phantomsection%
  \addcontentsline{toc}{subsection}{4.3. Fiabilité}%
  \label{fiabilite}%
}

On doit éviter toute intervention humaine vis à vis du système. Il doit disposer de mécanismes garantissant leur fonctionnement en continu.
Les aquisitions faites par les capteurs des cuves doivent être sures. Il conviendra donc d'effectuer une série de mesures pour déterminer la valeur réelle de la mesure captée et donc d'éviter la prises en compte de valeurs extraordinaires.


%___________________________________________________________________________

\subsection*{4.4. Évolutivité et maintenabilité%
  \phantomsection%
  \addcontentsline{toc}{subsection}{4.4. Évolutivité et maintenabilité}%
  \label{evolutivite-et-maintenabilite}%
}

Le système peut être amené à évoluer au niveau dimentionnel, fonctionnel et matériel. Il faut donc veiller à l'organiser en sous-systèmes, au comportement normalisé, et qui pourront donc être ajoutés, modifiés, ou retirés par la suite sans impacter le système global. Il faut définir les couches lors du développement de l'application pour faire en sorte que le système soit le plus indépendant possible du matériel. Le système devra être conçu pour supporter une mise à l'échelle européenne, que ce l'augmentation du flux de données, des besoins en termes de stockage ou dans la généricité de ses applications.


%___________________________________________________________________________

\subsection*{4.5. Limitations technologiques%
  \phantomsection%
  \addcontentsline{toc}{subsection}{4.5. Limitations technologiques}%
  \label{limitations-technologiques}%
}

Le système repose sur certaines technologies dont nous ne maîtrisons pas le fonctionnement (par exemple, le GSM ou le GPS). Il faut donc considérer la différence entre les technologies internes et externes mais surtout travailler les relations avec les prestataires responsables de ces technologies.
De plus, afin d'éviter ces problèmes de maîtrise, les équipements seront choisis chez des fournisseurs de confiance et les technologies devront généralement être open source avec une forte communauté. L'objectif est de garder un maximum de contrôle tout en évitant de voir la technologie et donc son support disparaître.


%___________________________________________________________________________

\subsection*{4.6. Généricité%
  \phantomsection%
  \addcontentsline{toc}{subsection}{4.6. Généricité}%
  \label{genericite}%
}

Notre système doit être conçu à pouvoir décliner à moindre coût pour d'autre applications de type surveillance. Les applications sont nombreuses, mais le fonctionnement global est souvent proche. L'organisation de notre solution doit donc faire abstraction du domaine d'application. Un grand effort sera mené au niveau du paramétrage global du système afin de garantir une extensibilité flexible et ne nécessitant qu'un travail de surface.
Les données provenant des mesures effectuées par les capteurs seront uniformisées au plus tôt, dès le noeud esclave pour garantir une communication de données homogènes à travers le réseau interne et pour éviter les impacts aux couches supérieurs : noeud maître et site central. Il en va de même pour les communications longue-distance, celles-ci devront utiliser des protocoles standards et des structures de données génériques. Le site central et les stations distantes devront fournir des interfaces de communication définies et génériques, abstraites du système sous-jacent. Globalement le développement tiendra compte de ces contraintes par la systémisation d'une architecture en couches et en modules interchangeables. A visée européenne, le logiciel devra être disponible sous l'ensemble des langues des pays concernés.


%___________________________________________________________________________

\subsection*{4.7. Réutilisation%
  \phantomsection%
  \addcontentsline{toc}{subsection}{4.7. Réutilisation}%
  \label{reutilisation}%
}

Les techniques employées ne sont pas fondamentalement nouvelles, nous pouvons donc en profiter pour réutiliser des composants (matériel, logiciels, libraries, patterns, bonnes pratiques, etc.) qui ont fait leurs preuves. Symétriquement, il faut que les composants de notre propre système puissent être réutilisés à leur tour. Le développement logiciel sera découpé en sous-projets mais cependant un effort en amont sera mené afin de garder une cohérence et une réutilisation à travers les différents sous-projets.


%___________________________________________________________________________

\subsection*{4.8. Ergonomie%
  \phantomsection%
  \addcontentsline{toc}{subsection}{4.8. Ergonomie}%
  \label{ergonomie}%
}

Le système s'adresse à des acteurs plus ou moins habitués à travailler avec l'informatique (opérationnels, camionneurs, télésurveillants...). Il faut donc que l'interface s'adapte à l'utilisateur et qu'il soit facile et convivial d'exploiter le système. Le besoin d'apprentissage doit être minimisé. Il faut aussi assurer la qualité de l'interface sur un appareil portable de type PDA, afin de faciliter son utilisation sur le terrain.


%___________________________________________________________________________

\subsection*{4.9. Traçabilité%
  \phantomsection%
  \addcontentsline{toc}{subsection}{4.9. Traçabilité}%
  \label{tracabilite}%
}

Le serveur central, qui est en communication avec l'ensemble du système, doit garder une trace de toutes les informations qui transitent et les états de fonctionnement de chaque station. Il en va de même pour l'enregistrement de toutes les opérations de maintenance. Du côté des stations, les systèmes embarqués doivent également assurer une traçabilité indépendante du serveur en cas de communications impossible ou d'erreurs graves. L'idée est d'offrir une possibilité de remonter dans l'historique des opérations dans le cas d'erreurs mais aussi dans le but d'analyser ces données (cf. FO-4-d).


%___________________________________________________________________________

\newpage \section*{Bilan des améliorations%
  \phantomsection%
  \addcontentsline{toc}{section}{Bilan des améliorations}%
  \label{bilan-des-ameliorations}%
}

Au final le nouveau système présente l'avantage de répondre en grande partie aux axes d'amélioration selectionnés en collaboration avec COPEVUE. Il répondra donc aux exigences de centralisation de la gestion des sites, de réduction des coûts et d'augmentation de la qualité du monitoring.

\end{document}
