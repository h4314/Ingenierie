\chapter{Présentation du rôle de CdP}

\textbf{tl;dr}

\section{Introduction}

Le rôle du chef de projet est d'animer le projet et de diriger l'équipe. Il
doit permettre de réduire le flou encadrant les étapes \textit{à venir} du
projet.

C'est bien pour le chef de projet de connaître son équipe (centres d'intérêts).
ORLY?

\section{Macrophasage}

$\rightarrow$ Approche temporelle (chronologique) : Grosso-modo (et je n'ai
peut-être pas tout compris) c'est le principe de faire un découpage en
"taches". (ORLY?)

\begin{enumerate}
	\item Dossier de faisabilité
	\item Cacher des charges (= spécifs ?)
	\item Conception
	\item Dossier de synthèse
\end{enumerate}

\section{Objectifs à court terme}

Le chef de projet doit être en mesure d'établir le programme de la première
séance.

\begin{itemize}
	\item Sonder le travail réalisé par chacun des membres,
	\item Prévoir l'organisation de la séance, la communiquer à l'équipe avant
la séance.
	\item S'assurer que tout le monde à pris connaissance du projet,
	\item La première séance incluera une réunion de 15-30 minutes avec R.A.
pour "faire un point".
	\item Faire "un point" avec l'équipe en fin de projet, et foutre
gentillement la pression.
\end{itemize}

Le dossier d'initialisation sera complété pendant la première séance (il doit
-à priori- déjà être bien avancé avant la première séance).

La première séance doit être consacrée à l'étude de faisabilité. 75\% des
questions d'étude de faisabilité doivent être réglées en fin de première
séance. 

\section{Saloperies courantes}

\begin{itemize}
	\item Un "GEI" peut être un branleur qui pose des questions à la con et
fait comme si il ne comprennait rien pour perdre du temps et rien foutre, la
réponse magique est (la plupart du temps) : "Régis et Marian sont à côté, prend
un ticket et attend en regardant dans le vague d'un air nostalgique".
	\item Il est possible (probable/systématique) que les différents profs
donnent des indications/réponses contradictoires, les faire se confronter, pour
vérifier les erreurs d'interprétations et Cie.
	\item Mauvaise gestion du temps $\rightarrow$ aller voir les profs pour
"trouver une solution" (lol?).
\end{itemize}

Il y aura régulièrement des "revues intermédiaires", pour vérifier
l'avancement, la cohérence, et cie. Elle doit être préparée :

\begin{itemize}
	\item la préparer avec un ordre du jour
	\item TOUT LE MONDE doit se forcer à acheter du terrain, ça fait bien et ça
force à faire un point sur l'état d'avancement de chacun
	\item QUOTD : "On ne sait rien mais on peut tout faire".
\end{itemize}

\subsection{Autorité du CdP ?}
% TODO ASK !
Ne pas hésiter à envoyer chier les gens, être détesté (surtout en début de
nouvel hexanôme, ça tombe bien), et dire à celui qui ne branle rien : "Tg, tu
finis pour telle date".

\section{Fin de la première periode}

Première période se terminant à la fin de la troisième séance.

\section{Dossier d'initialisation}

cf. \textit{Procédure de rédaction d'un dossier d'initialisation}. Les
documents sont également disponibles sur $Servif-baie$.

Le dossier d'initialisation est le résultat de la phase de lancement du projet. 

$\rightarrow$ \textbf{On met quoi dedans ???} % TODO !!!

\section{Procédure}

Draft intégré au travail de deuxième partie (PNP ? Wtf is this), en fait je
sais toujours pas qui c'est.

\section{Dossier de faisabilité}

On peut utiliser le plan type d'un dossier de faisabilité ($\rightarrow$
$Servif-baie$).

\begin{enumerate}
	\item Étude de l'existant (entre 0,5 et 1 page), $\rightarrow$ demander un
\textit{draft} pour la séance 1,
	\item Faisabilité :
	\begin{enumerate}
		\item Capteurs $\rightarrow$ faire une recherche sur internet, par
exemple 2h (limiter à des points principaux),
		\item Système embarqué (cartes) $\rightarrow$ est-il nécessaire de
faire notre propre électronique ? Que peut-on utiliser ? (idem 2h). Remarquons
que les normes et conditions en Norvege ne sont pas les mêmes qu'en France, et
c'est moins rigolo (ils faut en tenir compte). Il faut peut-être regarder des
cartes ou produits plus coûteux.
		\item OS utilisé (RTOS ?). Ce choix doit être
cohérent avec les exigences non fonctionnelles (prix, taille, caractèristiques
?) $\rightarrow$ Demander un tableau de comparaison des OS (ou un petit document, avec eventuelles annexes), globalement, ça peut être fait par celui qui fait des recherches sur les aspects embarqués.
		\item OS client (Gnu/Linux Ou Windows Ou AnotherShit ?)
		\item Production de l'énergie ? Rester écolo, respecter les
contraintes non-fonctionnelles ? $\rightarrow$ ça c'est la tâche achat de
terrain, éviter les trolls, tuer Daphnée P. \textbf{On veut un draft finalisé
sur les sources d'énergies} : qu'il arrête ses merdes, qu'il se concentre sur
l'essentiel et qu'il ne raconte pas trop de la merde (Daphnée Ing.).
		\item Systèmes de communication : sur sites isolés, et sur lieu d'étude
(client) $\rightarrow$ Demander un bon gros draft bien presque finalisé.
Attention, les outils de communication ne seront pas les mêmes.
	\end{enumerate}
	\item 
\end{enumerate}

Faire un relevé de décision : deux personnes vont se charger de reprendre le
travail d'étude de faisabilité pour produire le dossier 1.

Pendant ce temps, les deux autres travaillent sur le \textit{cahier des
charges} et produisent un \textit{Draft}. Quand c'est fait (pour les deux
équipes), passer à la conception. Le dossier de conception (dossier 3) doit
être finalisé en fin de 3e séance pour être rendu aux enseignants. Le dossier
de conception n'est pas non plus un truc méta-précis : l'objectif est de
montrer que ça marche (possiblement). Il faut bien comprendre que tout ce
travail peut n'avoir servi à rien du tout, puisqu'on peut (et que se sera le
cas) perdre l'appel d'offres.

\section{Best practices}

Les best-pratices doivent être rédigées pendant le projet, il faut le mettre à
jour très régulièrement.

\begin{description}
	\item[BP1] Best Practice 1 : aide à la rédaction
\end{description}

\section{Dossier de synthèse}

Grosso modo, le but c'est de rédiger une saleté de dossier (RQ+GEI) qui résume
à mort le résultat, pour faire lire à un mec omniscient qui connait déjà le
contenu.

\section{Approches}
\begin{enumerate}
	\item Chronologique
	\item Organisationnelle
	\item Produit
	\item Activité (+ revues internes et externes)
\end{enumerate}
